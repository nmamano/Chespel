\section{The compiler}

The compiler of \Chespel\ has been designed using ANTLRv3 and Java.
The compiler is installed by executing a \texttt{``make''} command
in the root directory of the \Chespel\ distribution. The user might need to
redefine the variables \texttt{ANTLR\_ROOT} and \texttt{JAVA\_LIBS}
to specify the location of ANTLR and Java libraries.

The compiler is invoked by executing the command \texttt{Chespel <file>}.
The compiler can also be invoked with some options that can be
listed by running \texttt{Chespel -help}:

\begin{verbatim}
$ Chespel -help
usage: Chespel [options] file
 -ast <file>     write the AST
 -dot            dump the AST in dot format
 -help           print this message
 -noexec         do not make the translation to C++
 -trace <file>   write a trace of function calls during the execution of
                 the program
\end{verbatim}

The option \texttt{-ast} generates a textual representation of the Abstract
Syntax Tree (AST). If the \texttt{-dot} option is used, the
representation is in dot format, that can be later transformed into
some graphical format using \texttt{dot}, e.g.:

\begin{verbatim}
$ Chespel example.chp -ast example.dot -dot -noexec
$ dot -Tpdf -o example.pdf example.dot
\end{verbatim}

The option \texttt{-noexec} must be used when the user does not want
to generate the resulting C++ file. It is useful when only a visualization of the AST is
required.

% Figure~\ref{fig:ast} depicts the AST of the first example used in this
% document.

% \begin{figure}
%  \centering
%  \includegraphics[width=\linewidth]{./fact.pdf}
%  % fact.pdf: 867x528 pixel, 72dpi, 30.59x18.63 cm, bb=0 0 867 528
%  \caption{AST of the factorial example.}
%  \label{fig:ast}
% \end{figure}

The option \texttt{-trace} can be used for debugging.
% For example, let us
% consider the following program that generates the moves for the problem of
% Hanoi towers.

% \begin{verbatim}
%  1:    func main()
%  2:      write "Number of disks: "; read n;
%  3:      nmoves = 0;
%  4:      hanoi(n, 1, 2, 3, nmoves);
%  5:      write "Total number of moves: "; write nmoves; write "%n";
%  6:    endfunc
%  7:
%  8:    func hanoi(n, from, aux, to, &mov)
%  9:      if n > 0 then
% 10:        hanoi(n-1, from, aux, to, mov);
% 11:        write "Move from "; write from;
% 12:        write " to "; write to; write "%n";
% 13:        mov = mov + 1;
% 14:        hanoi(n-1, aux, to, from, mov);
% 15:      endif
% 16:    endfunc
% \end{verbatim}

% In this example, the parameter \texttt{mov} is passed by reference and
% accumulates the number of moves. The rods are represented by the numbers 1, 2
% and 3.

% The execution the program with 3 disks would look as follows:

% \begin{verbatim}
%   $ Chespel hanoi.chp -trace hanoi.trace 
%   Number of disks: 3
%   Move from 1 to 3
%   Move from 1 to 3
%   Move from 2 to 1
%   Move from 1 to 3
%   Move from 2 to 1
%   Move from 2 to 1
%   Move from 3 to 2
%   Total number of moves: 7
% \end{verbatim}

% The file \texttt{hanoi.trace} reports the trace of function calls
% and returns with the values of the parameters and returned data.
% It also reports the value of the variables passed by reference
% at the return of each function. The contents of the file is the following:

% \begin{verbatim}
% main() <entry point>
% |   hanoi(n=3, from=1, aux=2, to=3, &mov=0) <line 4>
% |   |   hanoi(n=2, from=1, aux=2, to=3, &mov=0) <line 10>
% |   |   |   hanoi(n=1, from=1, aux=2, to=3, &mov=0) <line 10>
% |   |   |   |   hanoi(n=0, from=1, aux=2, to=3, &mov=0) <line 10>
% |   |   |   |   return, &mov=0 <line 9>
% |   |   |   |   hanoi(n=0, from=2, aux=3, to=1, &mov=1) <line 14>
% |   |   |   |   return, &mov=1 <line 9>
% |   |   |   return, &mov=1 <line 9>
% |   |   |   hanoi(n=1, from=2, aux=3, to=1, &mov=2) <line 14>
% |   |   |   |   hanoi(n=0, from=2, aux=3, to=1, &mov=2) <line 10>
% |   |   |   |   return, &mov=2 <line 9>
% |   |   |   |   hanoi(n=0, from=3, aux=1, to=2, &mov=3) <line 14>
% |   |   |   |   return, &mov=3 <line 9>
% |   |   |   return, &mov=3 <line 9>
% |   |   return, &mov=3 <line 9>
% |   |   hanoi(n=2, from=2, aux=3, to=1, &mov=4) <line 14>
% |   |   |   hanoi(n=1, from=2, aux=3, to=1, &mov=4) <line 10>
% |   |   |   |   hanoi(n=0, from=2, aux=3, to=1, &mov=4) <line 10>
% |   |   |   |   return, &mov=4 <line 9>
% |   |   |   |   hanoi(n=0, from=3, aux=1, to=2, &mov=5) <line 14>
% |   |   |   |   return, &mov=5 <line 9>
% |   |   |   return, &mov=5 <line 9>
% |   |   |   hanoi(n=1, from=3, aux=1, to=2, &mov=6) <line 14>
% |   |   |   |   hanoi(n=0, from=3, aux=1, to=2, &mov=6) <line 10>
% |   |   |   |   return, &mov=6 <line 9>
% |   |   |   |   hanoi(n=0, from=1, aux=2, to=3, &mov=7) <line 14>
% |   |   |   |   return, &mov=7 <line 9>
% |   |   |   return, &mov=7 <line 9>
% |   |   return, &mov=7 <line 9>
% |   return, &mov=7 <line 9>
% return <line 5>
% \end{verbatim}

\subsection{Organization of the compiler}

\begin{figure}
\Tree [.\fbox{\textbf{src}} 
[.\fbox{\textbf{Chespel}} Chespel.java ]
[.\fbox{\textbf{parser}} 
  Chespel.g\\\emph{ChespelLexer.java}\\\emph{ChespelParser.java}\\\emph{Chespel.tokens} ]
[.\fbox{\textbf{interp}} 
  Interp.java\\Data.java\\Memory.java\\ChespelTree.java\\ChespelTreeAdaptor.java ]
]
\caption{Structure of the \texttt{src} directory containing the source files
of the compiler.}
\label{fig:src}
\end{figure}

The source files of the compiler are organized in three packages
(see Fig.~\ref{fig:src}):
\begin{description}
 \item [Chespel:] contains the class \texttt{Chespel} with the entry point
(\texttt{main}) of the program. It calls the \emph{parser} and
\emph{translator}.
\item [parser:] contains the grammar and creator of the AST.
In the original distribution, this directory only contains the file
\texttt{Chespel.g}. The other files (italics in Fig.~\ref{fig:src}) are
created by ANTLR.
\item [interp:] contains the translator of the AST organized in different
files:
\begin{itemize}
 \item \emph{Interp.java}. It contains the core of the translator,
traversing the AST and executing the instructions.
\item \emph{Data.java}. It contains the class to represent data values
(integer and Boolean) and execute operations on them.
\item \emph{Memory.java}. It implements the memory of the translator with a
stack of activation records that contain pairs of strings (variable names)
and data (values).
\item \emph{ChespelTree.java}. It contains a subclass of the AST that extends the
information included in every AST node.
\item \emph{ChespelTreeAdaptor.java}. It contains a subclass required by ANTLR to
have access to the extended AST.
\end{itemize}
\end{description}

\subsubsection{Interp.java}

It has two public functions:
\begin{itemize}
 \item \texttt{Interp(ChespelTree T)}. This is the constructor of the
translator. It prepares the AST for the interpretation. It also creates a
table that maps function names into AST nodes (the roots of the functions).
The parameter \texttt{T} is the root of the AST.

\item \texttt{void Run()}. It runs the program by interpreting the AST.
\end{itemize}

Inside this file, there are some essential functions that constitute the
core of the translator. They execute function calls and instructions and
evaluate expressions.

\begin{itemize}
 \item \texttt{Data executeFunction(String funcname, ChespelTree args)}. This
function executes a function with a given name in which \texttt{args}
contains the AST of the list of arguments passed by the caller. The function
also implements the parameter passing mechanism.

\item \texttt{Data executeInstruction(ChespelTree t)}. This function executes
an individual instruction. In case the instruction returns some data (i.e. a
\texttt{return} instruction or some block of instructions that contains a
\texttt{return}), this function returns the data. Otherwise, a \texttt{null}
is returned.

\item \texttt{Data evaluateExpression(ChespelTree t)}. This function returns the
value produced by the evaluation of the expression represented by an AST.
\end{itemize}

During the execution of a program, the translator performs various semantic
and execution checks (access to undefined variables, operations with
incompatible types, division by zero, etc.) that may raise runtime exceptions.
The translator also keeps track of the line number of the source program for
an appropriate debugging when an exception is produced.

\subsubsection{Data.java}

This class represents a data item stored in memory. It provides methods to
retrieve the type and value of the data and perform arithmetic, logical and
relational operations. It also contains methods for semantic and runtime
checks.

\subsubsection{Memory.java}

This class represents the memory of the virtual machine that interprets
the program. \Chespel\ does not have global variables, thus all the visible
variables are local. Data can only be shared through the pass of parameters
or return of results. Parameters can be pass by value or reference.

\subsection{ChespelTree.java and ChespelTreeAdaptor.java}

These two files contain two classes that are required to extend the
information of the AST nodes. ANTLR allows this extension by creating
a subclass of the default class to represent AST nodes (\texttt{CommonTree}).

The extension implemented in \Chespel\ is very simple and allows to store the
integer and Boolean literals as Java \texttt{int}'s. This avoids the burden
to parse a text representation each time the literal needs to be read.

The class \texttt{ChespelTree} can be further extended for other purposes.
